\documentclass[fontset=windows]{ctexart}
\usepackage{fancyhdr}
\usepackage{xcolor}
\usepackage{listings}
\usepackage{geometry}
\usepackage{graphicx}
\usepackage{multirow}

\newcommand{\includecode}[2][]{%
    \lstinputlisting[#1]{#2}%
}

\geometry{a4paper, left=2.5cm, right=2.5cm, top=3cm, bottom=3cm}

% 设置页眉
\pagestyle{fancy}
\fancyhf{}
\fancyhead[L]{AIGC 创新赛}  % 左侧显示章节标题
\fancyhead[R]{FuFu.AI}  % 右侧自定义文字
\renewcommand{\headrulewidth}{0.4pt}

% 定义代码块样式
\lstset{
    backgroundcolor=\color{gray!10},
    basicstyle=\small\ttfamily,
    frame=single,
    numbers=left,
    numberstyle=\tiny\color{gray},
    breaklines=true,
    tabsize=4
}

\begin{document}

% 独立标题页
\begin{titlepage}
    \thispagestyle{empty}
    \vspace*{1cm}
    \centering
    \begin{figure}[h]
        \small
        \centering
        \includegraphics[width=0.4\textwidth]{./figure/LOGO.png}
    \end{figure}
    \vspace*{1cm}
    {\LARGE\bfseries IA Calendar 技术文档 \\[1.5em]}
    {\large Team: FuFu.AI \\[2em]}
    \vspace{1cm}
    {\normalsize
        \begin{tabular}{@{}ll@{}}
            上官子涵 & 西北工业大学 \\
            钟习伟  & 湖南大学   \\
            吴思贤  & 华南理工大学 \\
            何泓儒  & 华南理工大学 \\
            杨锦龙  & 华南理工大学 \\[2cm]
        \end{tabular}}

    \vfill
\end{titlepage}

\tableofcontents
\newpage{}

% 重置页面样式
\clearpage
\pagestyle{fancy}
\setcounter{page}{1}

\vspace{2cm}

\section{AI建议日程的实现}
\subsection{向AI发送请求}
以下的代码展示了在应用中是如何创建请求,并接收AI的响应的。
\includecode[language=java,caption={与蓝心大模型交互代码}]{src/code1.java}

\newpage{}
\subsection{AI响应的处理}
在获取AI的响应后,我们需要对其进行处理,以便将结果展示给用户。以下代码展示了如何解析AI的响应,并将其转换为可视化的日历事件。

此外,为应对可能出现的各种时间格式,设置了多种情况的处理逻辑,以确保AI给出的日程安排能够被正确解析和展示。
\includecode[language=java,caption={日程解析代码}]{src/code2.java}
\newpage{}

\section{日历的操作与显示}
\subsection{绘制空白单元格}
在日历中,我们需要绘制空白的单元格以便于用户查看和后续在单元格上绘制日期与当日事件。以下代码展示了如何在日历中绘制空白单元格。
\includecode[language=java,caption={绘制空白单元格代码}]{src/code3.java}
\newpage{}

\subsection{单元格事件的显示}
在日历中,我们需要将已添加的日程事件显示在对应的单元格中。

以下代码展示了如何从$SharedPreferences$加载有效事件,并将事件绘制到日历的单元格中。
\includecode[language=java,caption={单元格事件显示代码}]{src/code4.java}

在日历界面中,使用如下xml代码绘制日历网格:
\includecode[language=xml,caption={网格绘制}]{src/code5.xml}
\newpage{}

\section{今日日程与建议}
\subsection{当日日程的获取}
以下的代码展示了如何刷新今日日程,加载今日的事件并更新 RecyclerView 的数据。同时更新任务数据汇总和进度显示。
\includecode[language=java,caption={刷新今日日程的代码}]{src/code7.java}
\newpage{}
\subsection{获取近七天日程的建议}
在应用中,我们需要获取近七天的日程安排,并将其发送给AI以获取建议。以下代码展示了获取近七天的日程,并将其发送给AI进行处理。

此外,每日零点均会自动刷新获取AI建议的闹钟,以保证建议始终最新。
\includecode[language=java,caption={获取日程建议的交互代码}]{src/code6.java}
\newpage{}

\subsection{页面可视化}
\includecode[language=xml,caption={绘制今日日程页面}]{src/fragment_today.xml}
\end{document}